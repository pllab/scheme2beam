\section{Discussion}
\label{sec:discussion}

% Trade-offs in implementation language. 
The choice to use OCaml as our source compiler language resulted in several trade-offs. 
Erlang appears to be a natural choice for writing a source compiler for a language to Erlang's VM. 
Indeed, many languages that run on the BEAM are written in Erlang \cite{}. 
The biggest advantage here is that the source compiler does not need to specify and reimplement Core Erlang. 
Language designers can directly use the Core Erlang source code files from the Erlang compiler in their implementation. 
From a language runtime perspective, this choice also simplifies the runtime environment. 
End users only rely on the Erlang ecosystem for development---versus having an additional dependency on OCaml as is the case for our source compiler.

On the other hand, we chose OCaml because it is a language we are both familiar with---especially for writing compilers, interpreters, and program analyses. 
Given the time frame for the project, we felt we would be most productive using a language we are more familiar with, rather than adding more ramp-up time to learn how to write a compiler in Erlang. 
From a learning perspective, having to specify and implement the abstract and concrete representations of Core Erlang in OCaml taught us a lot about the internals of Erlang. 
This was an unexpected benefit. 
