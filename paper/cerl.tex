\section{The Erlang Compiler}
\label{sec:erlang-compiler}

At a high level, the flow of the Erlang compiler---from source code to BEAM bytecode---is as follows (see Figure \ref{fig:compiler-flow}). 
Erlang source code is first transformed to the Erlang \emph{abstract format}. 
The abstract format is close to the original source code, just represented as an abstract syntax tree. The compiler uses this format for front-end passes. 
From here, the Erlang abstract format is converted to \emph{Core Erlang}. 
Core Erlang is an intermediate representation (IR) used by the compiler for optimizations and other program analyses. 
See Section \ref{sec:core-erlang} for more discussion on Core Erlang. 
After the Erlang compiler performs optimizations and other program transformations over Core Erlang, it finally compiles down to BEAM bytecode. 

\section{Core Erlang as a Target Language}
\label{sec:core-erlang}

Core Erlang is a kind of subset of Erlang. 
Indeed, most high-level language constructs in Erlang source code compile down to sequences of function definitions and applications, case statements, guards, and let bindings in Core Erlang. 
Being a smaller language (but still higher level than BEAM bytecode) makes Core Erlang an ideal IR for program analyses. 
This also makes it an ideal target for our source compiler from Scheme. 
Other language designers have noted this as well. 
LFE (Lisp Flavored Erlang) \cite{} also compiles its source language to Core Erlang. 

Core Erlang has two syntactic representations: \emph{concrete} and \emph{abstract}. 
For actually performing program transformations over Core Erlang, the compiler uses the abstract representation, which specifies Core Erlang as data type over all syntactic keywords and constructs in the language. 
The Erlang compiler optionally emits concrete Core Erlang that is human readable. 

From this, the back end of our source compiler needs to handle both representations of Core Erlang. 
We need the abstract syntax to map parsed Scheme to constructs in Core Erlang. 
Then, we need to emit concrete Core Erlang that we feed into the Erlang compiler to produce BEAM bytecode. 

As an example, Figure \ref{fig:factorial} shows a factorial function written in Scheme and how it gets translated to concrete Core Erlang. 

\begin{figure}[!h]
\begin{lstlisting}[caption={Factorial function written in Scheme.},captionpos=b,frame=single,xleftmargin=1em]
(define (factorial n)
  (if (<= n 0)
      1
      (* n (factorial (- n 1)))))
\end{lstlisting}
\begin{lstlisting}[caption={Scheme factorial function translated to Core Erlang.},captionpos=b,frame=single,xleftmargin=1em]
module 'quick' ['factorial'/1]
  attributes []
'factorial'/1 =
  fun (_n) ->
    case <> of 
      <> when call 'erlang':'<='(_n,0) ->
        1
      <> when 'true' ->
        call 'erlang':'*'(_n,apply 'factorial'/1(call 'erlang':'-'(_n,1)))
    end
end
\end{lstlisting}
\label{fig:factorial}
\end{figure}

\subsection{An Abstract Core Erlang Specification for OCaml}

There is no official specification of Core Erlang.
The only reliable reference is the latest source code in the Erlang compiler. 
We relied on this closely for specifying Core Erlang in OCaml. 

There are about 30 syntactic constructs that make up Core Erlang. 
Since we are writing a source compiler for Scheme, we actually did not need to implement every single construct, only the ones necessary for expressing Scheme.
In the end, we ended up implementing 20 of the 30 language constructs. 

Implementation-wise, this is where OCaml shines. 
The Core Erlang specification is easily expressed as an algebraic data type in OCaml. 

\subsection{Generating Concrete Core Erlang}

To generate Core Erlang, we relied mostly on header comments in the Core Erlang source code, and our own hand-written Erlang examples that we compiled down to Core Erlang. 
Although not perfect, this informed us how to format concrete Core Erlang in order to be accepted by the Erlang compiler. 

Given that there is no formal specification of Core Erlang, we were limited in how we verified the correctness of generated Core Erlang. 
The best assurances we have is that our unit tests are accepted by the Erlang compiler, successfully compile down to BEAM, and can be run on the BEAM. 
